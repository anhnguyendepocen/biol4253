\documentclass[12pt]{article}
\usepackage{geometry}
\geometry{a4paper}
\usepackage[round]{natbib}
\usepackage{graphicx}
\usepackage[T1]{fontenc}
\usepackage[utf8]{inputenc}
\usepackage{textcomp}
\usepackage{gensymb}
\usepackage{amsmath}
\usepackage{amssymb}
\usepackage{authblk}
\usepackage[running]{lineno}
\usepackage{setspace}
\usepackage{rotating}
\usepackage{hyperref}

\setlength{\parindent}{0pt}


\usepackage{fancyhdr}
 
\pagestyle{fancy}
\fancyhf{}
\rhead{Tad Dallas (Biol 4253)}
\lhead{ How does global environmental change influence ecology?}

\doublespacing


\begin{document}











\subsection*{Reading:}

Camill, P. (2010) Global Change. Nature Education Knowledge 3(10):49 \\
\url{https://www.nature.com/scitable/knowledge/library/global-change-an-overview-13255365}

\bigskip

Walther, G.R., Post, E., Convey, P., Menzel, A., Parmesan, C., Beebee, T.J., Fromentin, J.M., Hoegh-Guldberg, O. and Bairlein, F., 2002. Ecological responses to recent climate change. Nature, 416(6879), p.389. \\ \url{http://eebweb.arizona.edu/courses/Ecol206/Walther%20et%20al%20Nature%202002.pdf}
















\begin{center}
\noindent\hrulefill 
\end{center}



\clearpage





\subsection*{How does global environmental change influence ecology?}
Much of the material we've covered so far has focused on theory and/or the influence of a static environment. That is, the niche was defined in terms of abiotic tolerances, but when we projected the niche into geographic space, we considered the environmental niche axis to take on a single value over a reasonable timeline. However, ecological systems are changing, some more rapidly than others. Before we go any further into this topic, I will note that I am not attempting to pander to any political or social group. Much of what we talk about today touches on issues that are treated as partisan, which I really don't agree with, but we will not discuss the political or social implications. Instead we will focus on the science, and specifically on how land use and environmental change affects ecological systems across all scales of organization (individual -- ecosystem). 

So we can start by operationalizing what we mean by \textit{land use change} and \textit{environmental change}. Here, we'll focus largely on human-induced changes to the landscape. 




\textit{Land use change} includes activities like logging, road/city development, and agriculture. That is, the physical land is changed from some (semi)-natural state to something pretty unnatural. This has direct effects on individual survival and subsequent population growth, as well as the potential to shift community structure when species differ in their sensitivity to disturbance.



\textit{Environmental change} refers to longer timescale changes to ecological systems, sometimes as a result of an indirect effect. This, in my definition, encompasses both \textit{climate change}, and secondary effects of pollution (e.g., acid rain). Climate change refers to the longterm increase in mean annual temperature, and the associated phenomena relating to an increase in extreme weather events (e.g., floods). 












\bigskip
\subsection*{Individual}

Land use change affects individuals by increasing dispersal distance and likelihood due to direct disturbance. Also, direct impacts on resource availability (both in positive and negative ways). 

**Discuss food supplementation in birds**



Environmental change influence individuals by differentially influence individual survival and adaptation responses. A good example of this is the melanization response in moths. 

**Discuss melanization response in moths**














\bigskip
\subsection*{Population}

Land use change influences populations by ...
- making some previously good habitat into not good habitat
- Fragmentation effects (reducing gene flow, recall Levins' model. Even removal of unoccupied patches reduces the equilibrium patch occupancy). 
- 
- 
- 

**Discuss **




Environmental change influence populations by ...
- shifting potential distribution of species. Let's discuss some niche evolution (adapt, move, or die) 
- Increasing disease prevalence (warmer temperatures may mean more disease) 
- 
- 

**Discuss range shifts**
















\bigskip
\subsection*{Community}

Land use change influences communities by ...
- Preferentially selecting for weedy species or urban exploiters (dandelions and clover wouldn't dominate most grassland communities)
- 
- 
- 

**Discuss **



Environmental change influence communities by ...
- Pheonological shifts (plant-pollinator mismatch can fuck up both parties)
- 
- 
- 

**Discuss **





















\bigskip
\subsection*{Ecosystems}

What is an ecosystem? What are the units of ecosystems? Why do we measure things like this, NPP shit?


Land use change influences ecosystems by ...
- removing them (draining of wetland habitats)
- 
- 
- 
-

**Discuss **






Environmental change influence ecosystems by ...
- 
- 
- 
- 


**Discuss **













\bigskip
\subsection*{Conclusions}














\bigskip
\subsection*{List of actionable items}

Waste less, eat less meat, drive less, fly less, buy less. 






\end{document}
